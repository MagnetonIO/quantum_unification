\section{Coherence for the Functorial Bridge}

Having introduced the functor
\(F\colon\mathcal Q\to\mathcal E\)
between quantum and classical worlds, we must show that it interacts
\emph{coherently} with all additional structure present in
$\mathcal Q$ and $\mathcal E$.  

Concretely this means:
\begin{enumerate}
  \item compatibility with the \emph{monoidal} ($\otimes$) structure;
  \item preservation, up to specified 2‑cells, of the
        \emph{dagger} (or $*$) symmetric structure modelling adjoints;
  \item functorial behaviour with respect to \emph{dynamics}, i.e.\
        time‑evolution 1‑parameter groups;
  \item compatibility with \emph{gauge} 2‑groupoids and their higher
        morphisms.
\end{enumerate}

We collect these requirements into a single coherence theorem.

\vspace{0.5cm}
\subsection{Setup and Notation}

Let
\((\mathcal Q,\otimes,\mathbb I,(-)^{\dagger})\)
be a \emph{dagger‑symmetric monoidal 2‑category}
whose 0‑cells are quantum systems, 1‑cells are
unitary channels (or path‑integral kernels), and 2‑cells are
intertwiners.

Let
\((\mathcal E,\times,\ast)\)
be a cartesian closed 2‑category (usually the topos of smooth
sheaves) modelling classical configuration spaces with smooth maps and
homotopies.

\begin{definition}[Lax Monoidal Dagger Functor]
A 2‑functor
\(F\colon\mathcal Q\to\mathcal E\)
is \emph{lax monoidal dagger} if it is equipped with 2‑natural
transformations
\(\phi_{A,B}\colon F(A)\times F(B)\Rightarrow F(A\otimes B)\)
and
\(\phi_{0}\colon \ast\Rightarrow F(\mathbb I)\)
such that:
\begin{align}
  \label{eq:pentagon}
  &\phi_{A,B\otimes C}\circ
    (\mathrm{id}\times\phi_{B,C}) \;\;\cong\;\;
    \phi_{A\otimes B,C}\circ(\phi_{A,B}\times\mathrm{id}),\\[4pt]
  \label{eq:unit}
  &\phi_{A,\mathbb I}\circ(\mathrm{id}\times\phi_{0})
    \;\;\cong\;\;
    \mathrm{id}_{F(A)} 
    \;\;\cong\;\;
    \phi_{\mathbb I,A}\circ(\phi_{0}\times\mathrm{id}),
\end{align}
and in addition
\(F(f^{\dagger}) \;=\; F(f)^{\dagger}\)
for every 1‑cell \(f\) in $\mathcal Q$.
\end{definition}

\vspace{0.5cm}
\subsection{The Coherence Theorem}

\begin{theorem}[Global Coherence]\label{thm:coherence}
The functor
\(F\colon\mathcal Q\to\mathcal E\)
constructed in Section~\ref{sec:framework}
admits a unique choice of coherence data
\((\phi_{-,-},\phi_{0})\)
making it a lax monoidal dagger functor.

Moreover, these data satisfy:
\begin{enumerate}
  \item \emph{Time‑evolution coherence}: for each Hamiltonian
        $H\in\mathcal Q(A,A)$ the diagram
        \[
        \begin{tikzcd}[row sep=large, column sep=large]
          F(A) \ar[r,"\exp(t\{H\,-\})"] \ar[d,swap,"\phi_{0}^{-1}\circ(-)"] &
          F(A) \ar[d,"\phi_{0}^{-1}\circ(-)"]\\
          F(A\otimes\mathbb I) \ar[r,swap,"F(\exp(-it H/\hbar))"] &
          F(A\otimes\mathbb I)
        \end{tikzcd}
        \]
        2‑commutes for all \(t\in\mathbb R\).
        
  \item \emph{Gauge‑compatibility}: if $G$ is a compact Lie
        2‑group acting on $A\in\mathcal Q$ then
        \(F\) factors through the homotopy fixed‑point 2‑category
        \(\mathcal E^{BG}\), and the induced functor
        respects 2‑morphisms coming from ghost insertions.
\end{enumerate}
\end{theorem}

\vspace{0.3cm}
\begin{proof}[Sketch]
Uniqueness follows from the fact that
$F$ restricts on objects to the classical‑limit map
$A\mapsto \operatorname{Spec}Z(\mathcal A_{A})$,
whose cartesian product is the only natural binary operation in
$\mathcal E$.

Existence: define
\(
  \phi_{A,B}\colon (x,p)\times(y,q)\mapsto
  \bigl(F_{A}(x,p),F_{B}(y,q)\bigr)
\)
and note that under $\hbar\to0$ the Baker–Campbell–Hausdorff series
truncates to the ordinary product.

Pentagon~\eqref{eq:pentagon} is a routine
monoidal‑category calculation;
unit laws~\eqref{eq:unit} are immediate by inspection.
The dagger condition uses that $\dagger$ is sent to
pointwise complex conjugation.

For (i) one combines Stone's theorem
$\exp(-itH/\hbar)=\mathrm{U}_{t}$ with the Poisson‑flow
$\exp\bigl(t\{-F(H),-\}\bigr)$ and checks equality of their
Hamiltonian vector fields.

For (ii) we use that BRST reduction commutes with $F$ and that
gauge 1‑ and 2‑morphisms act by conjugation, which becomes identity
after passing to gauge‑invariant functions in $\mathcal E$.
\end{proof}

\vspace{0.5cm}
\subsection{Strictification and 2‑Categorical Aspects}

Although $F$ is only \emph{lax} monoidal,
Mac Lane's strictification implies every diagram constructed from
\(
  \{\phi_{-,-},\phi_{0},F,\otimes,\times\}
\)
commutes up to a unique higher‑isomorphism.
Thus in computations one may \emph{pretend} $F$ is strict without loss
of generality, greatly simplifying graphical calculus.

\vspace{0.3cm}
\paragraph{Coherence diagrams.}
Throughout the paper we depict coherence via tikz‑cd squares and
hexagons, e.g.
\[
\begin{tikzcd}
F(A)\times F(B)\times F(C)
  \ar[r,"\phi_{A,B}\times\mathrm{id}"] \ar[d,swap,"\mathrm{id}\times\phi_{B,C}"]
&
F(A\otimes B)\times F(C)
  \ar[d,"\phi_{A\otimes B,C}"]
\\
F(A)\times F(B\otimes C)
  \ar[r,swap,"\phi_{A,B\otimes C}"]
&
F(A\otimes B\otimes C)
\end{tikzcd}
\]
whose commutativity is Eq.~\eqref{eq:pentagon}.

\vspace{0.5cm}
\subsection{Physical Implications}

Coherence guarantees the following:
\begin{itemize}
  \item \textbf{Consistent Classical Limits.}
        Any composite quantum process has a well‑defined
        classical shadow independent of parenthesisation or operator
        ordering, reinstating the usual classical‑physics intuition.
        
  \item \textbf{Gauge‑Invariant Emergence.}
        Non‑abelian charge conservation and Gauss‑law constraints
        commute with the $\hbar\to0$ limit, so that
        physical observables remain gauge‑invariant after functorial
        translation.
        
  \item \textbf{Functorial Dynamics.}
        Commutativity of the time‑evolution square makes $F$
        a \emph{symplecto‑functor}: it intertwines the quantum and
        classical Hamiltonian flows, thereby validating the path‑integral
        stationary‑phase rationale at a categorical level.
        
  \item \textbf{Higher‑Symmetry Compatibility.}
        The theorem extends to global 2‑symmetries (categorical groups),
        ensuring that e.g.\ topological phases with anyon content
        reduce consistently to classical bundle data plus Aharonov–Bohm
        holonomy.
\end{itemize}

\vspace{0.5cm}
\subsection{Outlook}

Coherence theorems like \ref{thm:coherence} are stepping stones toward a
\emph{homotopy‑coherent quantisation} functor
\[ \mathcal Q_\infty \;\longrightarrow\; \mathcal E_\infty, \]
from an $\infty$‑category of extended quantum field theories to the
$\infty$‑topos of smooth stacks.

Such a lift would capture anomalies, dualities, and
categorical states / operators in TQFTs on an equal footing.
We leave this generalisation to future work.