\section{Functorial Unification: Mapping Quantum Categories to Classical Topoi}\label{sec:functorial-mapping}
With the mathematical toolkit established, we now outline the core framework of our functorial approach to unification. The key idea is to represent quantum physics and classical physics within two structured categories (or a category and a topos, as we will clarify), and then connect them via a functor that preserves the relevant structures (such as composition of processes and logical relations). In broad strokes, we will consider a category $\mathcal{Q}$ that encapsulates the \textbf{quantum realm} and a topos (or category) $\mathcal{E}$ that encapsulates the \textbf{classical realm}, and construct a functor:
F: \mathcal{Q} \to \mathcal{E},
which we can think of as the \emph{unification functor} or \emph{semantics functor} translating quantum structures into classical or geometric ones.

In this section, we first describe typical choices for $\mathcal{Q}$ and $\mathcal{E}$ and what they signify physically (Subsection 3.1). Then, we define the functor $F$ concretely in a general setting (Subsection 3.2), discussing how states, observables, and symmetries are mapped. Subsection 3.3 explores properties of this functor, such as preserving composition (ensured by functoriality) and, when applicable, preserving the tensor product structure (so that $F$ is a monoidal functor) and the $*$-structure (so that $F(f^\dagger) = F(f)^\dagger$ in the classical context, where dagger in the classical topos is trivial or corresponds to taking the same map). We also discuss the possibility of an adjoint functor in the reverse direction, which would correspond to a quantization functor, establishing a pair of adjoint functors between quantum and classical domains (this is more speculative but conceptually appealing).

\subsection{Quantum Categories and Classical Topoi as Physical Universes}
\paragraph{Quantum category $\mathcal{Q}$:} For the quantum side, one convenient choice of $\mathcal{Q}$ is a category whose objects represent quantum systems (or perhaps contexts within a quantum system) and whose morphisms represent physical processes or relationships. There are several possibilities, depending on which aspect of quantum theory we want to emphasize:
	•	We can take $\mathcal{Q}$ to be the category of $C^$-algebras (or von Neumann algebras) and $$-homomorphisms. In this picture, each object is an algebra of observables of some quantum system, and a morphism $\varphi: A \to B$ could be an embedding of one system’s observables into another (for instance, an inclusion of $A$ as a subalgebra of $B$, or a homomorphism corresponding to a quantum channel or simulation).
	•	Alternatively, we might take $\mathcal{Q}$ to be a category of state spaces: e.g. finite-dimensional Hilbert spaces (with linear maps, or completely positive maps, between them). This is natural if one thinks of an object as a Hilbert space describing a quantum system, and a morphism as a physical process or evolution from one state space to another. However, one might then restrict to isometric or unitary maps for reversible processes, or consider decorated morphisms for general quantum operations (CP maps).
	•	A more structured approach is to consider a \emph{dagger compact category} of quantum processes, like the category whose objects are finite-dimensional Hilbert spaces and morphisms are not just linear maps but equivalence classes of circuits or diagrams generated by some gate set. This approach is taken in categorical quantum mechanics, where one might have additional generators for preparation and measurement, but for our purposes the simpler description in terms of linear maps suffices as all finite-dimensional processes can be represented in that way.
	•	In some contexts, $\mathcal{Q}$ could be a 2-category or higher category capturing quantum fields (with objects as spacetimes, 1-morphisms as fields, and 2-morphisms as transformations between fields, etc.), but this is quite advanced and we will mostly stick to the quantum mechanics or quantum algebra perspective in our examples.

Throughout this paper, we will often implicitly work with a $\mathcal{Q}$ that is a dagger monoidal category to embody the idea that multiple systems can be combined ($\otimes$) and that processes have adjoints ($\dagger$). For instance, one concrete setting to keep in mind is $\mathcal{Q} = \mathbf{FdHilb}$, the category of finite-dimensional Hilbert spaces (objects) and linear maps (morphisms), which is a dagger symmetric monoidal category. This $\mathcal{Q}$ is rich enough to describe finite quantum systems, including composite ones (via tensor product) and probabilistic amplitudes (with dagger giving the inner product structure). However, the framework is not limited to finite dimensions; one could consider infinite-dimensional Hilbert spaces or algebraic quantum theory as needed, though that introduces additional technicalities (topology, etc.).

\paragraph{Classical category/topos $\mathcal{E}$:} For the classical side, we typically use a topos or a related category that captures classical states or spacetime. Some choices are:
	•	$\mathcal{E} = \mathbf{Set}$ or perhaps $\mathbf{Set}^{\mathcal{C}^{op}}$ (a presheaf category). Using $\mathbf{Set}$ itself corresponds to a very naive embedding of quantum into classical: effectively one would assign to each quantum observable a set of possible values (its spectrum). More generally, a presheaf topos allows context dependence; as mentioned, if $\mathcal{C}$ is a category of contexts (such as abelian subalgebras of a quantum algebra), $\widehat{\mathcal{C}} = [\mathcal{C}^{op}, \mathbf{Set}]$ is a topos in which one can construct a more nuanced classical representation of the quantum system. Indeed, as referenced, the topos of contravariant functors on the poset of commutative subalgebras contains the \emph{spectral presheaf} of a noncommutative algebra, which is a central object in the topos approach to quantum physics .
	•	$\mathcal{E}$ could also be the category of sets \textit{with additional structure}, such as $\mathbf{Meas}$ (sets with a sigma-algebra, i.e., measurable spaces) or $\mathbf{Diff}$ (smooth manifolds and smooth maps). Often classical phase spaces are smooth manifolds or symplectic manifolds, so one might consider $\mathcal{E}$ to be a category of manifolds or symplectic manifolds with structure-preserving maps. However, manifolds are not a topos (they lack certain colimits globally), though one can embed them into a sheaf topos as seen in Example~\ref{ex:Sh-topos}.
	•	A very relevant option is to let $\mathcal{E}$ be a topos of sheaves on a spacetime manifold, i.e. $\mathcal{E} = \mathbf{Sh}(M)$. In this topos, an object can be seen as a varying set (or structure) over spacetime, which is a good way to treat classical fields: a classical field configuration can be identified with a global section of a sheaf, and the collection of all such sections (for all open sets) forms a sheaf object. Using a topos of sheaves means our classical context includes the idea of localization in spacetime. This is important if we want to unify with general relativity, since general relativity can be thought of as a theory of the spacetime manifold and fields on it.
	•	Another classical topos that has been advocated in the literature of quantum foundations is the topos of presheaves on the lattice of commutative subalgebras of a quantum algebra . This is a contravariant Set-functor category on that poset. In such a topos, the spectral presheaf $\Sigma$ (an object that assigns to each context its spectrum) plays the role of the state space (a kind of generalized phase space) of the quantum system. This approach by Döring and Isham essentially attaches a topos $\mathcal{E}$ to each quantum system, rather than having one fixed $\mathcal{E}$ for all; we can incorporate that by considering a functor $F$ that depends on the system or by making $\mathcal{E}$ big enough to include multiple systems as separate connected components.

For clarity, in this paper we often think of $\mathcal{E}$ as either $\mathbf{Set}$ or a presheaf topos $\widehat{\mathcal{C}}$ for some context category $\mathcal{C}$. The simplest case to illustrate ideas is $\mathcal{E}=\mathbf{Set}$. But ultimately, using a richer $\mathcal{E}$ (like a sheaf or presheaf topos) is necessary for capturing phenomena like varying classical frames or emergent classicality from quantum. When unifying with gravity, one might take $\mathcal{E}$ to even be a topos that encodes different possible spacetimes or geometries (like a topos of sheaves on the category of all small manifolds, which is indeed a topos). For specificity in examples, we mostly use $\mathbf{Set}$ or simple presheaves, but the formalism supports more exotic choices of $\mathcal{E}$.

\paragraph{States and Observables:} In both categories $\mathcal{Q}$ and $\mathcal{E}$, one can identify objects that correspond to the notion of a “state space” and morphisms that correspond to “observables” or “transformations”. For example, in $\mathcal{Q}$ (say $\mathbf{FdHilb}$), an object $H$ is a state space, and a (bounded linear) morphism $f: H \to K$ that is self-adjoint and idempotent (a projector) can represent a yes-no observable (a proposition) in that system. In $\mathcal{E}$ (say $\mathbf{Set}$ or $\mathbf{Sh}(M)$), an object $X$ might represent a space of states, and a morphism $\chi: X \to \Omega$ in the topos is a truth-valued function on $X$, analogous to a predicate or property of states (like “the particle is in region $U$”). The functor $F: \mathcal{Q}\to\mathcal{E}$ will be constructed such that to each quantum observable or property we can associate a classical observable or property in $\mathcal{E}$. For example,
F(\hat{x}) = x, \quad F(\hat{p}) = p, \quad F(\hat{H}) = H(x,p),
for a particle with position $\hat{x}$, momentum $\hat{p}$, and Hamiltonian $\hat{H} = \hat{p}^2/2m + V(\hat{x})$ mapping to the classical position coordinate $x$, momentum $p$, and Hamiltonian function $H(x,p) = p^2/2m + V(x)$. We will see this explicitly in Section4. More formally, if $\mathcal{Q}$ has a commutative subcategory for each context (like the category of a single commutative algebra $A$ with its spectral decomposition), then $F$ restricted to that subcategory acts like the Gelfand spectrum functor, mapping the algebra $A$ to its set $\Sigma(A)$ of maximal ideals (spectrum) in $\mathcal{E}$ , and each $*$-homomorphism between contexts to the corresponding continuous map between spectra. That is indeed how one constructs the spectral presheaf: $A \mapsto \Sigma(A)$. Our approach just doesn’t fix one context but tries to relate all contexts at once via the topos.

\subsection{Explicit Construction of the Functor $F$}
We now describe how one can explicitly define a functor $F: \mathcal{Q} \to \mathcal{E}$ linking the quantum and classical structures. The guiding principle is that $F$ should send each quantum system to a classical avatar (often a set or space of “states” or “points” that correspond to classical configurations), and each quantum process or relation to a corresponding classical process or relation.

One canonical example of such a functor arises from the Gelfand spectrum. If $\mathcal{Q}$ is the category of commutative $C^$-algebras (which, by the Gelfand–Naimark theorem, are dual to locally compact Hausdorff spaces), one can define a contravariant functor $\Sigma: \mathcal{Q} \to \mathbf{LocSp}$ (category of localic spaces, or essentially topological spaces) by sending each commutative $C^$-algebra $A$ to its spectrum $\Sigma(A)$ (the space of characters, which are homomorphisms $A\to \mathbb{C}$) . On morphisms: a $C^*$-homomorphism $\phi: A \to B$ (between commutative algebras) induces a continuous map $\Sigma(\phi): \Sigma(B) \to \Sigma(A)$ (pullback of characters). This is a contravariant functor (opposite direction on morphisms), but one can equivalently consider covariant functors on the opposite category. For simplicity, we might think in terms of the opposite category of commutative algebras, which is (under Gelfand duality) equivalent to the category of locally compact Hausdorff spaces. Thus, there is a duality functor that is actually a contravariant equivalence. The main takeaway: there’s a functorial correspondence between classical state spaces and abelian quantum algebras.

For \emph{noncommutative} algebras (general quantum systems), one cannot assign a single space as its spectrum (the algebra does not have characters separating points if it is noncommutative). However, one can proceed with the idea of a \textbf{contextual spectrum}: consider the category $\mathcal{C}(A)$ of all commutative subalgebras (contexts) of a noncommutative algebra $A$ (with inclusion maps as morphisms). For each such context $C \subseteq A$, one can find its spectrum $\Sigma(C)$ which is a classical space (topologically, a compact Hausdorff space if $C$ is unital abelian $C^*$). Now define a presheaf $X_A$ on the context category $\mathcal{C}(A)$ by $C \mapsto \Sigma(C)$. This assignment is contravariantly functorial in $C$, meaning if $C_1 \subseteq C_2$ then there is a restriction map $\Sigma(C_2) \to \Sigma(C_1)$ (since $\Sigma$ is contravariant on homomorphisms $C_1 \hookrightarrow C_2$). The collection of these spaces ${\Sigma(C)}$ with restriction maps forms a presheaf (in fact a sheaf if we give the category $\mathcal{C}(A)$ an appropriate Grothendieck topology). This presheaf is known in quantum topos literature as the \textbf{spectral presheaf} of $A$ .

The spectral presheaf $X_A$ is an object in the presheaf topos $\widehat{\mathcal{C}(A)}$. However, to compare different algebras, we can fiber this construction. In fact, there is a way to see it as a functor $F$: choose $\mathcal{Q}$ as (noncommutative) $C^$-algebras, and $\mathcal{E}$ as the category of presheaves on commutative subalgebras (but this $\mathcal{E}$ depends on $A$ itself, so it’s not one fixed target category for all $A$ unless one takes a disjoint union of all those topoi). A more elegant approach is to consider the category of \emph{all pairs} $(A,C)$ where $A$ is a $C^$-algebra and $C$ a commutative subalgebra, and a suitable category of all pairs $(X,p)$ where $X$ is a topological space and $p$ a point in it, and then define a functor between those that essentially does $(A,C) \mapsto (\Sigma(C), \text{embedding})$. This becomes heavy; instead, let’s articulate $F$ in more physical terms:

For each quantum system (object) $Q$ in $\mathcal{Q}$, $F(Q)$ is a set or space of classical states of $Q$'' or classical snapshots of $Q$.’’ In practice, one often takes $F(Q)$ to be the set of all pure states of $Q$ (if $Q$ is described by a Hilbert space, pure states correspond to one-dimensional projections; if $Q$ is described by an algebra, pure states are extremal positive linear functionals). However, the set of pure states is not in general a good classical state space (for example, it might lack a simple manifold structure if $\dim H>1$ except as a projective space). Another choice is $F(Q) = \Sigma(\mathcal{A}_Q)$, the spectrum of a preferred maximal abelian subalgebra (MASA) of the system’s algebra $\mathcal{A}_Q$. But choosing a single context breaks symmetry.

The spectral presheaf circumvented the need to choose by taking all contexts at once, but then $F(Q)$ is a presheaf, not a set. That is acceptable since $\mathcal{E}$ could be a presheaf category. Indeed, one could set $\mathcal{E} = \mathbf{Set}^{\mathcal{C}^{op}}$ where $\mathcal{C}$ is the category of contexts of a “typical system”, $\widehat{\mathcal{C}}$ is a topos, and then a specific $Q$ yields a functor $\mathcal{C}(Q) \to \mathcal{C}$ which induces a morphism in that topos. This is a bit beyond scope, so instead we adopt a simpler viewpoint:
we will treat
F(Q)
as “the (generalized) state space of $Q$” in a classical sense. For example:
	•	If $Q$ is a quantum particle on $\mathbb{R}^3$, one might take $F(Q)$ to be the set of points in $\mathbb{R}^3$ (its classical configuration space) or better the phase space $\mathbb{R}^6$. But since quantum states also include momentum distribution, one might prefer the space of all probability distributions on phase space. However, that is too large (infinite-dimensional). Another approach in topos form is indeed to consider the spectral presheaf of the algebra of observables (which for a particle would assign to each maximal abelian subalgebra of $L^\infty(\mathbb{R}^3)$ its spectrum).
	•	If $Q$ is described by an algebra $\mathcal{A}$, perhaps not abelian, $F(Q)$ could be the spectral presheaf $\Sigma(_)$ on its context category, which has the property that its points (if any exist) correspond to two-valued homomorphisms on $\mathcal{A}$, essentially classical valuations of all observables at once (which typically do not exist for a genuinely quantum $\mathcal{A}$ due to Kochen–Specker). The absence of points is a reflection of quantum no-go theorems; in the topos approach one then works with $\Sigma$’s subobjects instead of points. But the functor $F$ can still produce $\Sigma$ itself as a classical avatar of $Q$ (a sort of phase space object albeit without points).

In less formal terms, the functor $F$ is defined by some prescription that to each quantum system $Q$ associates a classical state space $F(Q)$ (which might be a set, topological space, or a presheaf), and to each quantum morphism $f: Q_1 \to Q_2$ associates a function or mapping $F(f): F(Q_1) \to F(Q_2)$ that describes how a classical state of $Q_1$ transforms into a classical state of $Q_2$.

If $f$ is an inclusion of systems (like $Q_1$ is a subsystem of $Q_2$), $F(f)$ could be a surjection or projection of state spaces (like projecting a classical state of the bigger system down to a classical state of the subsystem by marginalizing over extra degrees of freedom). If $f$ is an irreversible process (like a quantum channel that irreversibly loses information), then $F(f)$ might be a multi-valued mapping or a stochastic map between classical state spaces (since a single initial classical state of $Q_1$ might correspond to a distribution of possible states of $Q_2$). In a simple case, consider $f$ a quantum measurement process that yields a classical outcome in some set $X$; then $F(f)$ could be a map from the prior classical state space to (the set of probability distributions on) $X$. One way to handle this is to allow $F(f)$ to land not in $\mathbf{Set}$ but in a category of sets and stochastic maps, or to treat it as a relation (making $\mathcal{E}$ something like $\mathbf{Rel}$). For clarity, we often consider reversible or inclusion maps where $F(f)$ can be defined as a single-valued function. In general, extending to relations or Markov kernels is possible but involves working in a different target category (e.g. $\mathbf{Stoch}$ instead of $\mathbf{Set}$).

Thus, we have a blueprint:
F: \mathcal{Q} \to \mathcal{E}, \quad Q \mapsto X_Q = F(Q), \quad (f: Q_1 \to Q_2) \mapsto (F(f): X_{Q_1} \to X_{Q_2})~,
with the property that for each state $x \in X_{Q_1}$ (a classical state of $Q_1$), $F(f)(x) = y$, where $y$ is a classical state of $Q_2$ such that for every observable $A$ of $Q_1$, $x(A) = y(f(A))$. In other words, $y$ restricted via $f$ to $Q_1$ yields $x$. This definition makes sense if $f$ is something like an embedding of $Q_1$’s observables into $Q_2$’s observables. Then $F(f)$ just says: any classical configuration of $Q_1$ can be extended to a classical configuration of $Q_2$ that agrees on the image of $Q_1$. If $f$ is surjective (like a projection of $Q_2$ onto $Q_1$), then $F(f)$ would be a restriction map: given a classical state of $Q_2$, $F(f)$ gives its classical state on $Q_1$ by forgetting the extra components. This is a partial inversion of a quantum channel scenario (here $f$ might be an inclusion of $\mathcal{A}{Q_1}$ into $\mathcal{A}{Q_2}$ forgetting an environment, then $F(f)$ restricts a full classical state to the subsystem state).

However, if $F(f)$ fails to be well-defined single-valued (like many extensions exist or many preimages), one might instead define $F(f)$ as a relation or a multi-valued function. This would still be a perfectly good morphism in a category of relations or stochastic maps rather than plain functions. It suggests that strictly using $\mathbf{Set}$ might be too narrow to capture all quantum processes, and indeed a more general $\mathcal{E}$ (like a topos of stochastic relations) might be needed to capture irreversible processes.

For now, we will often limit attention to either structure-preserving embeddings or expectation values in fixed states, so that $F(f)$ can be treated as a partial function or at least something manageable. The rigorous way to handle a general quantum channel $\mathcal{E}$ is to allow $\mathcal{E}$ to be something like $\mathbf{Stoch}$ (with objects as measurable spaces and morphisms as Markov kernels, which indeed form a category albeit not a topos). Since our focus is structural, we won’t delve further into this technical point here.

Thus, we define a functor
F: \mathcal{Q} \to \mathcal{E}
by specifying its action on objects and morphisms as above. The important point is that $F$ is required to satisfy $F(\mathrm{id}Q) = \mathrm{id}{F(Q)}$ and $F(g\circ f) = F(g)\circ F(f)$, which in physical terms means that doing two quantum processes $f$ then $g$ and then translating to classical effect is the same as translating each and composing (which is natural if the translation is correct). We design $F$ to ensure this: if $h = g\circ f$, then for a classical state $x$ of $Q_1$, $F(h)(x)$ should equal $F(g)(F(f)(x))$ because both represent the classical state of $Q_3$ after the combined process with initial state $x$. Ensuring this often requires using expectation values or distributions when processes are not deterministic.

One important property of $F$ is whether it preserves the monoidal structure. If $\mathcal{Q}$ is monoidal with $\otimes$ and $\mathcal{E}$ has a corresponding product (like $\times$ for state spaces), we expect that
F(Q_1 \otimes Q_2) \cong F(Q_1) \times F(Q_2),
i.e. the classical state space of a composite quantum system is (in leading order) the Cartesian product of the classical state spaces of the components. This holds in classical physics: the state of two independent systems is a pair $(x_1, x_2)$. In quantum physics, entangled states are not simple pairs, but in the classical limit (as $\hbar\to0$ or large quantum numbers) entanglement becomes negligible or can be described as classical correlations (which are distributions on the product space). Thus, $F$ should be a \textbf{monoidal functor}: $F(Q_1 \otimes Q_2) \cong F(Q_1)\otimes’ F(Q_2)$ and $F$ maps a pair of morphisms to the pair of their images. We will in examples simply assume $F$ respects composition of systems – indeed in our harmonic oscillator example, combining two oscillators $Q_1, Q_2$ to $Q_1\otimes Q_2$ corresponds under $F$ to combining two phase spaces $\mathbb{R}^2$ and $\mathbb{R}^2$ into $\mathbb{R}^4$ by direct product.

Additionally, $F$ might preserve the $*$-structure: meaning if $f: Q \to Q$ is a quantum symmetry (unitary), $F(f): F(Q) \to F(Q)$ should be a classical symmetry (perhaps a diffeomorphism or permutation of classical states). If $h: Q \to Q$ is a self-adjoint idempotent (projector), then $F(h)$ should be a characteristic function of a subset of $F(Q)$. These correspondences ensure that $F$ doesn’t destroy the probabilistic interpretation: e.g., $F(\hat{\rho}^\dagger) = F(\hat{\rho})^\dagger$ will make $\hat{\rho}$ Hermitian (density operator) correspond to $F(\hat{\rho})$ being a real measure on the classical space. In $\mathbf{Set}$, a dagger has no content beyond equality, but if we work in $\mathbf{Rel}$, the dagger is relation converse, which correspond to inverting cause and effect, something we might consider if $F$ maps a unitary to a bijective function so that it’s invertible ($F(f)^\dagger = F(f^{-1})$ in $\mathbf{Set}$ trivial since $f^{-1} = f^\dagger$ in $\mathbf{Hilb}$ and $F(f^{-1}) = F(f)^{-1}$ in $\mathbf{Set}$). So in effect, $F$ maps unitary to bijection and Hermitian to involution on set (like a reflection perhaps). We will not belabor the dagger in examples because classical categories often don’t have a nontrivial dagger (except $\mathbf{Rel}$ does), but implicitly we want $F$ to map symmetric structures to symmetric structures (which is part of preserving the interpretation that probabilities sum to 1, etc.).

As one of the major accomplishments of this framework, we effectively encode the \textbf{correspondence principle} into the functor: when a quantum concept has a classical analog, $F$ will map it appropriately, and as quantum parameters approach classical regimes, $F$ becomes essentially an equivalence between the quantum category and the classical category. We will demonstrate in examples that $F$ indeed recovers Hamilton’s equations from Heisenberg’s equations, and Poisson brackets from commutators, etc.

\subsection{Properties and Examples of the Functor $F$}
To cement the idea of $F$, let’s illustrate with simple examples and check functorial properties:
	•	\textbf{Trivial system:} Let $I_{\mathcal{Q}}$ be the unit object in $\mathcal{Q}$ (e.g. a 1-dimensional Hilbert space, representing a vacuum or a trivial system). We expect $F(I_{\mathcal{Q}}) = I_{\mathcal{E}}$, the unit in $\mathcal{E}$. In $\mathbf{Set}$, the unit object is a one-point set (terminal object). Indeed, a trivial quantum system has only one classical state (the trivial thing).
	•	\textbf{Composite systems:} For two quantum systems $Q_1, Q_2$, their composite is $Q_1 \otimes Q_2$. Now $F(Q_1 \otimes Q_2)$ should correspond to a classical description of both together. If $Q_1$ and $Q_2$ are independent, classically one would consider $F(Q_1) \times F(Q_2)$ as the state space of the pair. So we impose an isomorphism $F(Q_1 \otimes Q_2) \cong F(Q_1)\times F(Q_2)$ in $\mathcal{E}$. For example, if $Q_1$ is a qubit and $Q_2$ is another qubit, then $F(Q_i)$ might be something like ${0,1}$ (two possible values if we had a definite classical bit). Then $F(Q_1\otimes Q_2) \cong {0,1} \times {0,1} = {00, 01, 10, 11}$. We note that entangled quantum states do not correspond to single points in this Cartesian product (they correspond to probability distributions or lack of a joint assignment), but our framework’s $F$ would capture them via either a distribution on this set or by not being assignable to a single global element of the presheaf in a topos approach. For coherence’s sake, if we consider states that are well-approximated by product states (like coherent states for large oscillators), $F$ will carry them to points of the product space, and more generally to measures on it.
	•	\textbf{Identity morphism:} For any quantum system $Q$, $F(\mathrm{id}Q)$ should equal $\mathrm{id}{F(Q)}$. This is required by functor axioms and indeed is naturally satisfied by our definitions (the classical mapping of doing nothing is doing nothing).
	•	\textbf{Composition:} If $Q_1 \xrightarrow{f} Q_2 \xrightarrow{g} Q_3$ are two quantum morphisms (say $f$ is an inclusion of $Q_1$ into $Q_2$, $g$ an evolution from $Q_2$ to $Q_3$), then we need $F(g\circ f) = F(g)\circ F(f)$. In our interpretation: given a classical state $x$ of $Q_1$, if we first use $F(f)$ to view it as a state of $Q_2$, then $F(g)$ to view it as a state of $Q_3$, that’s the same as directly viewing $Q_1$’s state as a state of $Q_3$ via $g\circ f$. This holds if our definition of $F$ was via restriction and extension of state assignments as above, because restricting in one step or two steps yields the same result. If $f$ and $g$ are such that their composition loses no more information than individually (e.g. $f$ is embedding, $g$ is invertible or some projection and we track appropriate distributions), then indeed $F(g)\circ F(f)$ is well-defined and equals $F(g\circ f)$. In cases where a slight mismatch could occur (like if $f$ introduced some correlation and $g$ lost some info), that would reflect a subtlety: either $F(g\circ f)$ should be seen as a relation encompassing both effects at once (which is indeed equal to the relation composition of $F(g)$ and $F(f)$). The bottom line: the functor condition can be maintained if we define $F$ carefully, and we will implicitly ensure it in our examples.

One interesting property is whether $F$ has an adjoint, i.e. a functor $G: \mathcal{E} \to \mathcal{Q}$ such that $G \dashv F$ or $F \dashv G$ (depending on variance). A left adjoint $G$ would mean a kind of quantization functor: given a classical system $X$, $G(X)$ would be a quantum object whose classical shadow is $X$. For instance, for a symplectic manifold $M$, one might define $G(M)$ as an appropriate quantization Hilbert space or $C^*$-algebra. However, quantization is not a functor on all classical systems (because of ordering ambiguities, etc., as per Groenewold–van Hove no-go theorem). But in simple cases (like finite sets to finite-dimensional commutative algebras or discrete Wigner functions), one can have such an adjoint. While elegant, we won’t require an explicit adjoint functor $G$ in our framework; it suffices that one exists on a formal level (like an equivalence in the $\hbar\to0$ inverse limit). The presence of an adjoint would formalize Bohr’s correspondence: quantization ($G$) and classical limit ($F$) are inverses in some limit or category, meaning the unified theory truly covers both without loss. This remains a target for future development.

In the next section, we will apply this formal construction to specific systems to demonstrate how it works and to verify that no physical content is lost in translation and that known results are reproduced.