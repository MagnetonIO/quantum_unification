\section{Illustrative Examples of Functorial Unification}

To see the framework in action we spell out three canonical cases:
\begin{enumerate}
  \item the quantum \emph{harmonic oscillator};
  \item the \emph{Dirac field} (free spin‑$\tfrac12$ fermions);
  \item non‑abelian \emph{Yang–Mills} gauge theory.
\end{enumerate}

In each case we
\begin{itemize}
  \item construct a category (or algebra) $\mathcal Q$ encoding the quantum
        system;
  \item identify a classical topos or phase space $\mathcal E$;
  \item define a functor $F\colon\mathcal Q\to\mathcal E$;
  \item verify that $F$ reproduces classical dynamics when $\hbar\to0$.
\end{itemize}

\bigskip

\subsection{Quantum Harmonic Oscillator}\label{subsec:QHO}

Recall the Hamiltonian
\(
  \widehat H=\dfrac{\widehat p^{2}}{2m}+\dfrac12 m\omega^{2}\widehat x^{2},
\)
with $[\widehat x,\widehat p]=i\hbar$.  The algebra of observables
$\mathcal A_{\mathrm{HO}}$ is the Weyl–Heisenberg algebra; we regard the
(one‑object) category $\mathcal Q_{\mathrm{HO}}$ whose morphisms are
elements of $\mathcal A_{\mathrm{HO}}$.
The classical counterpart is the symplectic manifold
$\mathbb R^{2}=\{(x,p)\}$ with canonical form $\mathrm d x\wedge\mathrm d p$.

\medskip

\paragraph{Functor.}
Define $F\colon\mathcal Q_{\mathrm{HO}}\to\mathcal E_{\mathrm{HO}}$ by
\[
  F(\widehat x)=x,\qquad
  F(\widehat p)=p,\qquad
  F(\widehat H)=\dfrac{p^{2}}{2m}+\dfrac12 m\omega^{2}x^{2},
\]
and extend linearly.  $F$ is \emph{not faithful}: operators that differ by
ordering map to the same classical polynomial.  This loss of information is
precisely the collapse of quantum interference in the classical limit.

\medskip

\paragraph{Dynamics.}
In the Heisenberg picture
\(
  \dot{\widehat x}=\tfrac{i}{\hbar}[\widehat H,\widehat x]=\widehat p/m,
  \;
  \dot{\widehat p}=\tfrac{i}{\hbar}[\widehat H,\widehat p]=-m\omega^{2}\widehat x.
\)
Applying $F$ yields Hamilton's equations
\(
  \dot x=p/m,\;
  \dot p=-m\omega^{2} x,
\)
so $F$ interchanges quantum and classical time evolution.  Thus the functor
\emph{implements the correspondence principle}: as $\hbar\to0$, commutators
vanish and $\mathcal Q_{\mathrm{HO}}$ degenerates to $\mathcal E_{\mathrm{HO}}$.

\bigskip

\subsection{Dirac Field}\label{subsec:Dirac}

Let $\psi\colon\mathbb R^{1,3}\to\mathbb C^{4}$ be a spinor satisfying
\(
  (i\gamma^{\mu}\partial_{\mu}-m)\psi=0.
\)
The canonical equal‑time anticommutation relations generate the
\emph{CAR‑algebra} $\mathcal A_{\mathrm{D}}$, and we define
$\mathcal Q_{\mathrm{D}}=\operatorname{Rep}(\mathcal A_{\mathrm{D}})$, the
category of *‑representations on $\mathbb Z_{2}$‑graded Hilbert spaces.

\medskip

\paragraph{Classical topos.}
The classical data are Grassmann‑valued spinor fields
$(\psi,\bar\psi)$ on Minkowski space; the natural ambient topos is
$\mathbf{Sh}(\mathsf{Diff})$, the sheaf topos over smooth manifolds, enriched
by a line object of Grassmann numbers.  Denote this by $\mathcal E_{\mathrm{D}}$.

\medskip

\paragraph{Quantisation functor.}
Wick ordering followed by $\hbar\to0$ defines a functor
$F\colon\mathcal Q_{\mathrm{D}}\to\mathcal E_{\mathrm{D}}$ that maps operators
$\bar\psi\gamma^{\mu}\psi$ to the classical bilinears
$\bar\psi\gamma^{\mu}\psi$, and similarly for the stress–energy tensor.
In the limit $\hbar\to0$ (or equivalently large occupancy), the Schwinger–Dyson
equations reduce to the classical Dirac equation, recovered as
$F(\widehat O)\approx O_{\mathrm{cl}}$ for all composite operators
$\widehat O$.

\medskip

\paragraph{Spin–statistics check.}
Because $F$ preserves $\mathbb Z_{2}$‑grading, fermionic sign data are erased
only after we pass to global sections (i.e.\ ordinary functions), consistent
with the fact that Grassmann signs have no classical analogue.

\bigskip

\subsection{Yang–Mills Gauge Fields}\label{subsec:YM}

Let $G$ be a compact Lie group with Lie algebra $\mathfrak g$.
The quantum theory is defined by the algebra generated by gauge potentials
$\widehat A_{\mu}^{a}(x)$ and their conjugate momenta, modulo the Gauss‑law
constraint.  Formally this is the \emph{BRST‑reduced} algebra
$\mathcal A_{\mathrm{YM}}$.  We take
$\mathcal Q_{\mathrm{YM}}=\operatorname{Rep}(\mathcal A_{\mathrm{YM}})$.

\medskip

\paragraph{Classical topos.}
Classically a Yang–Mills configuration is a principal $G$‑bundle $P\to M$
with connection $A\in\Omega^{1}(P,\mathfrak g)$; the natural topos
$\mathcal E_{\mathrm{YM}}$ is the
slice topos $\mathbf{Sh}(\mathsf{Diff})/BG$, where $BG$ is the
smooth groupoid of $G$‑bundles.

\medskip

\paragraph{The functor $F$.}
On objects $F$ sends a quantum representation to its expectation‑value
connection
\(
  A_{\mu}^{a}(x)=\bigl\langle\widehat A_{\mu}^{a}(x)\bigr\rangle,
\)
and on morphisms (Wilson loops, field‑strength operators, ghost insertions)
it takes $\hbar\to0$ asymptotics.
At leading order in $\hbar$ the Yang–Mills Schwinger–Dyson hierarchy reduces
to the classical field equations $D^{\mu}F_{\mu\nu}=0$.
Non‑abelian holonomy is preserved because parallel transport is encoded
categorically as a functor
$\Pi_{1}(M)\to G\text{-}\mathbf{Set}$, and $F$ acts trivially on that level.

\medskip

\paragraph{Confinement \& topology.}
Quantum topological sectors labelled by instanton number map to distinct
components of $\mathbf{Sh}(\mathsf{Diff})/BG$; thus $F$ is
\emph{essentially surjective but not full}: several inequivalent quantum vacua
collapse to the same classical gauge potential, again reflecting loss of
quantum information.

\bigskip

These three examples demonstrate that the functorial machinery:
\begin{enumerate}
  \item reproduces the textbook classical limits;
  \item handles fermionic as well as bosonic degrees of freedom;
  \item respects gauge symmetry, including non‑trivial bundle topology.
\end{enumerate}

They thereby provide concrete evidence that our categorical approach is a
coherent unification mechanism across the spectrum of modern physics.