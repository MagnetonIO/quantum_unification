\section{Comparison with Existing Approaches}\label{sec:comparison}
Section~7 of the introduction sketched why a \emph{single lax monoidal
dagger functor}
\(
  F\colon \mathcal Q \longrightarrow \mathcal E
\)
is more than a cosmetic reformulation of the
quantum–classical correspondence.
Here we provide a detailed head‑to‑head comparison with
six major frameworks, summarised in Table~\ref{tab:comparison}.%
\footnote{We intentionally exclude semiclassical WKB techniques; those can be
viewed as local approximations inside our functorial calculus and are
analysed in Appendix~B.}

\begin{table}[ht]
\centering
\renewcommand{\arraystretch}{1.2}
\begin{tabular}{@{\quad}lcccccc@{\quad}}
\toprule
\textbf{Feature}
 & \textbf{F‑Unification}
 & \textbf{Can.\ Quant.}
 & \textbf{Geom.\ Quant.}
 & \textbf{AQFT}
 & \textbf{CQM}
 & \textbf{Topos‐Q}\\
\midrule
Globally functorial          & \checkmark & $\times$ & $\times$ & partial & partial & \checkmark\\
Handles higher symmetries    & \checkmark & partial  & partial  & \checkmark & partial & partial\\
Gauge‑invariant by design    & \checkmark & $\times$ & partial  & \checkmark & $\times$ & \checkmark\\
Classical limit internalised & \checkmark & external & external & external & external & internal\\
Supports $\infty$‑categories & \checkmark & $\times$ & $\times$ & $\times$ & partial & partial\\
AI‑assisted formalisation    & \checkmark & $\times$ & $\times$ & $\times$ & $\times$ & $\times$\\
\bottomrule
\end{tabular}
\caption{Qualitative feature matrix (✓ = inherent, partial = attainable with
extra work, $\times$ = missing).  Acronyms: Canonical Quantisation (Can.\ Quant.), Geometric Quantisation (Geom.\ Quant.), Algebraic QFT (AQFT), Categorical Quantum Mechanics (CQM), and Topos‑based Quantum Theory (Topos‑Q).\label{tab:comparison}}
\end{table}

\subsection{Canonical Quantisation}
The textbook recipe promotes
$\{x_i,p_j\}=\delta_{ij}$ \(\mapsto\) \([\widehat x_i,\widehat p_j]
=i\hbar\delta_{ij}\).
It operates \emph{object‑by‑object}, hence fails to be
\emph{functorial}: a canonical transformation in phase space need not lift to
a unitary map on Hilbert space.
By contrast, our functor $F$ makes the classical
category \emph{the image} of the quantum one, so any morphism in
$\mathcal Q$ automatically has a classical shadow.
Functoriality therefore upgrades the correspondence principle from a mere
heuristic to a structurally enforced equivalence at $\hbar\!=\!0$.

\subsection{Geometric Quantisation}
Geometric quantisation \cite{Kostant1970,Souriau1970} erects a polarised
Hilbert bundle over phase space and extracts wave‑functions as covariantly
constant sections.  
While powerful for integrable systems, it is sensitive to global anomalies
and offers no canonical prescription for interacting QFTs.  
Our framework absorbs polarisation dependence into the choice of 1‑cells of
$\mathcal Q$; changes of polarisation become explicit 2‑cells, so anomalies
manifest as obstructions to extending $F$ to a higher‑categorical
equivalence—see Remark \ref{rem:anomaly} in Section~\ref{sec:coherence}.

\subsection{Algebraic Quantum Field Theory}
AQFT encodes observables in a net
$O\mapsto\mathcal A(O)$ of local C$^{*}$‑algebras satisfying isotony and
Haag duality.
This excels at locality but obscures the classical limit
(\emph{contra} Haag–Swieca’s infrared problem).
Inside our picture the entire net is a single object of
$\mathcal Q$; locality survives as a factorisation system
\( \mathcal Q \xrightarrow{F} \mathcal E \hookrightarrow \mathbf{Sh}(\mathsf{Diff})\).
Hence ultraviolet regularity and infrared behaviour become
\emph{morphisms}—a strictly finer level of control.

\subsection{Categorical Quantum Mechanics}
Abramsky–Coecke’s diagrammatic CQM formalism
captures finite‑dimensional quantum processes via
dagger‑compact categories.
It is inherently \emph{topological}—
geometry and analysis enter only through ad hoc
scalars.
Our $\mathcal Q$ contains CQM as the 0‑semi‑simple sub‑2‑category generated
by dualisable objects; the extra analytic data live in higher morphisms,
allowing us to speak about e.g.\ unbounded operators and renormalisation
flows that lie outside conventional CQM.

\subsection{Topos‑Based Quantum Theory}
Isham, Butterfield, and Döring embed a quantum system into a
presheaf topos $\mathbf{Set}^{\mathcal V^{\operatorname{op}}}$
over its lattice of commutative subalgebras $\mathcal V$.
They achieve an \emph{internal} classical logic but still treat the
quantum–classical interface extrinsically.  
Our $F$ unifies the two logics inside a \emph{single}
arrow, revealing that the presheaf construction is the
right adjoint to a universal property satisfied by $F$
(Proposition 7.3).
Hence the topos approach appears as a special case of our functorial
bridge, not vice versa.

\subsection{Novel Contributions}
\paragraph{Higher‑Symmetry Readiness.}
Because $\mathcal Q$ and $\mathcal E$ are 2‑categories (extendable to
$\infty$‑categories), higher‑form and categorical symmetries integrate
seamlessly.  Conventional frameworks bolt these on
\emph{post facto}.

\paragraph{AI‑Assisted Proof Logistics.}
The formal definition of $F$ is short, but concrete calculations
generate diagrams with thousands of 2‑cells.  
We leverage LLM‑driven proof synthesis (see Section~9) to discharge
coherence conditions algorithmically—unexplored by previous approaches.

\paragraph{Minimal Classical Assumptions.}
Our construction requires only that $\mathcal E$ be cartesian closed and
locally presentable; no symplectic form or Poisson structure is demanded
\emph{a priori}.  Those appear \emph{a posteriori} as images of quantum
commutators, broadening the scope to dissipative and open systems.

\subsection{When Do Frameworks Coincide?}
In many simple models—finite‐dimensional Hilbert spaces, no gauge
symmetry, polynomial Hamiltonians—all six frameworks \emph{agree up to
equivalence}.  Differences emerge as soon as one introduces
\begin{itemize}
  \item higher bundles or gerbes (topological phases);
  \item infinite‑dimensional state spaces (QFT);
  \item non‑separable Hilbert spaces (quantum gravity candidates).
\end{itemize}
Precisely in these fertile regions does Functorial Unification retain
structural clarity while its competitors fragment into case‑by‑case fixes.

\subsection{Outlook}
The present comparison suggests a conjecture:

\begin{conjecture}[Universality]
Every reasonable quantisation functor
\( Q\colon\mathcal E'\to\mathcal Q' \)
factors (up to homotopy) through the lax monoidal dagger functor
\(F\) constructed here.
\end{conjecture}

Proving universality would elevate $F$ to the role of a
\emph{terminal object} in the $\infty$‑category of quantisation
procedures—a categorical crown jewel that would sharpen debates on
uniqueness of quantisation and the meaning of the classical limit.